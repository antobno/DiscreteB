\chapterimage{blue26.jpeg} % Imagen de encabezado de capítulo
\chapterspaceabove{6.75cm} % Espacio en blanco desde la parte superior de la página hasta el título del capítulo en las páginas del capítulo
\chapterspacebelow{7.25cm} % Cantidad de espacio en blanco vertical desde el margen superior hasta el comienzo del texto en las páginas de los capítulos

\chapter*{BIBLIOGRAFÍA}

\section*{Fuente básica}

\begin{enumerate}[label={[\arabic*]}]
    \item Grimaldi, R. P. (1998). Matemáticas Discretas y Combinatoria: Una introducción con aplicaciones (3a. ed., 3a. reimp.). México: Addison-Wesley Iberoamericana\footnote{Aunque existen tres fuentes básicas, para esta obra usaremos principalmente esta fuente.}.
    \item Robert R. Korfhage (1970). Lógica y Algoritmos. Limusa Wiley.
    \item Kenneth H. Rosen (1991). Discrete Mathematics and its applications (2a. ed.). McGraw-Hill.
\end{enumerate}

\section*{Fuente de consulta}

\begin{enumerate}[resume,label={[\arabic*]}]
    \item Abraham P. Hillman-Gerald L. Alexanderson-Richar M. Grass (1990). Discrete and Combinational Mathematics. Cullier Macmillan.
    \item McEliece-ASH (1989). Introduction to Discrete Mathematics. Rndom-House.
    \item Mauer-Ralston (1991). Discrete Algorithmic Mathematics. Addison-Wesley.
    \item Kolman-Busby (1997). Estructuras de Matemáticas Discretas para la Computación. Pearson Educación.
    \item Irving M. Copi (1979). Lógica Simbólica. C.E.C.S.A.
    \item H. Enderton (1972). A Mahematical Introduction to Logic. Academic Press.
    \item Michael Spivak (2000). Calculus (2a ed.). Reverté.
\end{enumerate}

\newpage
\pagestyle{empty}
\,