\chapterimage{blue19.jpeg} % Imagen de encabezado de capítulo
\chapterspaceabove{6.75cm} % Espacio en blanco desde la parte superior de la página hasta el título del capítulo en las páginas del capítulo
\chapterspacebelow{7.25cm} % Cantidad de espacio en blanco vertical desde el margen superior hasta el comienzo del texto en las páginas de los capítulos

%------------------------------------------------

\chapter{EL USO DE LA COMPUTADORA}


La computadora es una herramienta fundamental en la sociedad actual y tiene una amplia gama de aplicaciones en diversos campos, desde el ámbito empresarial y científico hasta el entretenimiento y la comunicación. El uso de la computadora ha revolucionado la forma en que trabajamos, nos comunicamos y nos entretenemos.



Uno de los aspectos clave en el uso de las computadoras es la programación. Los lenguajes de programación son conjuntos de instrucciones que permiten a los desarrolladores comunicarse con las computadoras y crear software y aplicaciones. Existen muchos lenguajes de programación diferentes, cada uno con sus propias características y aplicaciones específicas. Algunos de los lenguajes de programación más populares incluyen:
\begin{enumerate}
\item Python: Es un lenguaje de programación versátil y fácil de aprender, que se utiliza ampliamente en campos como la ciencia de datos, la inteligencia artificial y el desarrollo web.
\item Java: Es un lenguaje de programación de propósito general que se utiliza en una amplia variedad de aplicaciones, desde aplicaciones empresariales hasta desarrollo de Android.
\item C++: Es un lenguaje de programación de bajo nivel que se utiliza en el desarrollo de sistemas, juegos y aplicaciones de alto rendimiento.
\item JavaScript: Es un lenguaje de programación que se utiliza principalmente para desarrollar aplicaciones web interactivas y dinámicas.
%\item Ruby: Es un lenguaje de programación conocido por su elegancia y facilidad de lectura, utilizado principalmente en el desarrollo web.
\item C\#: Es un lenguaje de programación desarrollado por Microsoft, utilizado principalmente en el desarrollo de aplicaciones de Windows y juegos de Unity.
\item PHP: Es un lenguaje de programación utilizado principalmente en el desarrollo de aplicaciones web y la creación de sitios dinámicos.
\end{enumerate}


Estos son solo algunos ejemplos, y existen muchos otros lenguajes de programación disponibles, cada uno con sus propias ventajas y desventajas. La elección del lenguaje de programación dependerá de los requisitos del proyecto, las preferencias del desarrollador y las características específicas que se necesiten. Considerando que las computadoras son herramientas muy importantes en nuestra actualidad, particularmente, para ahorrarnos tiempo y esfuerzo en el trabajo de calcular, proporcionaremos algunos programas escritos en lenguaje C++, para hacer el cálculo que en cada caso se indica.

\newpage

\section*{Programa para evaluar un número en la función suelo y techo}

\noindent
\begin{lstlisting}[language=C++]
#include <iostream>
#include <cstdlib>
#include <locale.h>
#include <cmath>

int main() {
    double numero;
    std::cout << "Introduce un numero real: ";
    std::cin >> numero;

    double suelo = floor(numero);
    double techo = ceil(numero);

    std::cout << "El suelo de " << numero << " es: " << suelo << std::endl;
    std::cout << "El techo de " << numero << " es: " << techo << std::endl;

    system ("pause>null");
    return 0;
}
\end{lstlisting}

\section*{Programa que calcula las raíces de una ecuación de segundo grado}

\noindent
\begin{lstlisting}[language=C++]
#include <iostream>
#include <cstdlib>
#include <locale.h>
#include <cmath>

int main() {
    setlocale(LC_ALL, "");
    double a, b, c;
    double discriminante, raiz1, raiz2;

    std::cout << "Ingrese los coeficientes de la ecuacion de segundo grado:" << std::endl;
    std::cout << "a: ";
    std::cin >> a;
    std::cout << "b: ";
    std::cin >> b;
    std::cout << "c: ";
    std::cin >> c;

    discriminante = b * b - 4 * a * c;

    if (discriminante > 0) {
        raiz1 = (-b + sqrt(discriminante)) / (2 * a);
        raiz2 = (-b - sqrt(discriminante)) / (2 * a);

        std::cout << "Las raices son reales y diferentes:" << std::endl;
        std::cout << "Raiz 1: " << raiz1 << std::endl;
        std::cout << "Raiz 2: " << raiz2 << std::endl;
    } else if (discriminante == 0) {
        raiz1 = -b / (2 * a);

        std::cout << "Las raices son reales e iguales:" << std::endl;
        std::cout << "Raiz: " << raiz1 << std::endl;
    } else {
        double parteReal = -b / (2 * a);
        double parteImaginaria = sqrt(-discriminante) / (2 * a);

        std::cout << "Las raices son complejas conjugadas:" << std::endl;
        std::cout << "Raiz 1: " << parteReal << " + " << parteImaginaria << "i" << std::endl;
        std::cout << "Raiz 2: " << parteReal << " - " << parteImaginaria << "i" << std::endl;
    }

	system ("pause>null");
	return 0;
}

\end{lstlisting}

\section*{Programa que calcula la suma de los primeros $\bm{n}$ números naturales}

\noindent
\begin{lstlisting}[language=C++]
#include <iostream>
#include <cstdlib>
#include <locale.h>

int main() {
    setlocale(LC_ALL, "");
    int n;
    int suma = 0;

    std::cout << "Ingrese el valor de N: ";
    std::cin >> n;

    for (int i = 1; i <= n; i++) {
        suma += i;
    }

    std::cout << "La suma de los " << n << " primeros numeros naturales es: " << suma << std::endl;

	system ("pause>null");
	return 0;
}

\end{lstlisting}

\section*{Programa que calcula la suma de los primeros $\bm{n}$ números pares}

\noindent
\begin{lstlisting}[language=C++]
#include <iostream>
#include <cstdlib>
#include <locale.h>

int main() {
    setlocale(LC_ALL, "");
    int n;
    int suma = 0;

    std::cout << "Ingrese el valor de N: ";
    std::cin >> n;

    for (int i = 2; i <= n * 2; i += 2) {
        suma += i;
    }

    std::cout << "La suma de los " << n << " primeros numeros pares es: " << suma << std::endl;

	system ("pause>null");
	return 0;
}

\end{lstlisting}

\section*{Programa que verifica si un número es primo}

\noindent
\begin{lstlisting}[language=C++]
#include <iostream>
#include <cstdlib>
#include <locale.h>

bool esPrimo(int numero) {
    if (numero <= 1) {
        return false;
    }

    for (int i = 2; i * i <= numero; i++) {
        if (numero % i == 0) {
            return false;
        }
    }

    return true;
}

int main() {
    setlocale(LC_ALL, "");
    int numero;

    std::cout << "Ingrese un numero: ";
    std::cin >> numero;

    if (esPrimo(numero)) {
        std::cout << numero << " es un numero primo." << std::endl;
    } else {
        std::cout << numero << " no es un numero primo." << std::endl;
    }

	system ("pause>null");
	return 0;
}

\end{lstlisting}

\section*{Programa que calcula el MCD de dos números enteros}

\noindent
\begin{lstlisting}[language=C++]
#include <iostream>
#include <cstdlib>
#include <locale.h>

int calcularMCD(int num1, int num2) {
    while (num2 != 0) {
        int temp = num2;
        num2 = num1 % num2;
        num1 = temp;
    }

    return num1;
}

int main() {
    setlocale(LC_ALL, "");
    int num1, num2;

    std::cout << "Ingrese el primer numero: ";
    std::cin >> num1;

    std::cout << "Ingrese el segundo numero: ";
    std::cin >> num2;

    int mcd = calcularMCD(num1, num2);

    std::cout << "El Maximo Comun Divisor (MCD) de " << num1 << " y " << num2 << " es: " << mcd << std::endl;

	system ("pause>null");
	return 0;
}

\end{lstlisting}

\section*{Programa que suma dos matrices}

\noindent
\begin{lstlisting}[language=C++]
#include <iostream>
#include <cstdlib>
#include <locale.h>

const int MAX_FILAS = 100;
const int MAX_COLUMNAS = 100;

void sumarMatrices(int matriz1[][MAX_COLUMNAS], int matriz2[][MAX_COLUMNAS], int resultado[][MAX_COLUMNAS], int filas, int columnas) {
    for (int i = 0; i < filas; i++) {
        for (int j = 0; j < columnas; j++) {
            resultado[i][j] = matriz1[i][j] + matriz2[i][j];
        }
    }
}

int main() {
    setlocale(LC_ALL, "");
    int filas, columnas;
    int matriz1[MAX_FILAS][MAX_COLUMNAS];
    int matriz2[MAX_FILAS][MAX_COLUMNAS];
    int resultado[MAX_FILAS][MAX_COLUMNAS];

    std::cout << "Ingrese el numero de filas: ";
    std::cin >> filas;
    std::cout << "Ingrese el numero de columnas: ";
    std::cin >> columnas;

    std::cout << "Ingrese los elementos de la matriz 1:" << std::endl;
    for (int i = 0; i < filas; i++) {
        for (int j = 0; j < columnas; j++) {
            std::cout << "Ingrese el elemento " << i+1 << "," << j+1 << ": ";
            std::cin >> matriz1[i][j];
        }
    }

    std::cout << "Ingrese los elementos de la matriz 2:" << std::endl;
    for (int i = 0; i < filas; i++) {
        for (int j = 0; j < columnas; j++) {
            std::cout << "Ingrese el elemento " << i+1 << "," << j+1 << ": ";
            std::cin >> matriz2[i][j];
        }
    }

    sumarMatrices(matriz1, matriz2, resultado, filas, columnas);

    std::cout << "La matriz resultado es:" << std::endl;
    for (int i = 0; i < filas; i++) {
        for (int j = 0; j < columnas; j++) {
            std::cout << resultado[i][j] << " ";
        }
        std::cout << std::endl;
    }

	system ("pause>null");
	return 0;
}

\end{lstlisting}

\section*{Programa que multiplica dos matrices}

\noindent
\begin{lstlisting}[language=C++]
#include <iostream>
#include <cstdlib>
#include <locale.h>

const int MAX_FILAS = 100;
const int MAX_COLUMNAS = 100;

void multiplicarMatrices(int matriz1[][MAX_COLUMNAS], int matriz2[][MAX_COLUMNAS], int resultado[][MAX_COLUMNAS], int filas1, int columnas1, int columnas2) {
    for (int i = 0; i < filas1; i++) {
        for (int j = 0; j < columnas2; j++) {
            resultado[i][j] = 0;
            for (int k = 0; k < columnas1; k++) {
                resultado[i][j] += matriz1[i][k] * matriz2[k][j];
            }
        }
    }
}

int main() {
    setlocale(LC_ALL, "");
    int filas1, columnas1, filas2, columnas2;
    int matriz1[MAX_FILAS][MAX_COLUMNAS];
    int matriz2[MAX_FILAS][MAX_COLUMNAS];
    int resultado[MAX_FILAS][MAX_COLUMNAS];

    std::cout << "Ingrese el numero de filas de la matriz 1: ";
    std::cin >> filas1;
    std::cout << "Ingrese el numero de columnas de la matriz 1: ";
    std::cin >> columnas1;
    std::cout << "Ingrese el numero de filas de la matriz 2: ";
    std::cin >> filas2;
    std::cout << "Ingrese el numero de columnas de la matriz 2: ";
    std::cin >> columnas2;

    if (columnas1 != filas2) {
        std::cout << "Error: Las matrices no se pueden multiplicar." << std::endl;
        return 0;
    }

    std::cout << "Ingrese los elementos de la matriz 1:" << std::endl;
    for (int i = 0; i < filas1; i++) {
        for (int j = 0; j < columnas1; j++) {
            std::cout << "Ingrese el elemento " << i+1 << "," << j+1 << ": ";
            std::cin >> matriz1[i][j];
        }
    }

    std::cout << "Ingrese los elementos de la matriz 2:" << std::endl;
    for (int i = 0; i < filas2; i++) {
        for (int j = 0; j < columnas2; j++) {
            std::cout << "Ingrese el elemento " << i+1 << "," << j+1 << ": ";
            std::cin >> matriz2[i][j];
        }
    }

    multiplicarMatrices(matriz1, matriz2, resultado, filas1, columnas1, columnas2);

    std::cout << "La matriz resultante es:" << std::endl;
    for (int i = 0; i < filas1; i++) {
        for (int j = 0; j < columnas2; j++) {
            std::cout << resultado[i][j] << " ";
        }
        std::cout << std::endl;
    }

	system ("pause>null");
	return 0;
}

\end{lstlisting}

%\newpage

\section*{Programa que calcula la transpuesta de una matriz}

\noindent
\begin{lstlisting}[language=C++]
#include <iostream>
#include <cstdlib>
#include <locale.h>

const int MAX_FILAS = 100;
const int MAX_COLUMNAS = 100;

void obtenerMatrizTranspuesta(int matriz[MAX_FILAS][MAX_COLUMNAS], int filas, int columnas, int matrizTranspuesta[MAX_COLUMNAS][MAX_FILAS]) {
    for (int i = 0; i < filas; i++) {
        for (int j = 0; j < columnas; j++) {
            matrizTranspuesta[j][i] = matriz[i][j];
        }
    }
}

void mostrarMatriz(int matriz[MAX_FILAS][MAX_COLUMNAS], int filas, int columnas) {
    for (int i = 0; i < filas; i++) {
        for (int j = 0; j < columnas; j++) {
            std::cout << matriz[i][j] << " ";
        }
        std::cout << std::endl;
    }
}

int main() {
    setlocale(LC_ALL, "");
    int filas, columnas;
    int matriz[MAX_FILAS][MAX_COLUMNAS];
    int matrizTranspuesta[MAX_COLUMNAS][MAX_FILAS];

    std::cout << "Ingrese el numero de filas de la matriz: ";
    std::cin >> filas;
    std::cout << "Ingrese el numero de columnas de la matriz: ";
    std::cin >> columnas;

    std::cout << "Ingrese los elementos de la matriz:" << std::endl;
    for (int i = 0; i < filas; i++) {
        for (int j = 0; j < columnas; j++) {
            std::cin >> matriz[i][j];
        }
    }

    obtenerMatrizTranspuesta(matriz, filas, columnas, matrizTranspuesta);

    std::cout << "Matriz original:" << std::endl;
    mostrarMatriz(matriz, filas, columnas);

    std::cout << "Matriz transpuesta:" << std::endl;
    mostrarMatriz(matrizTranspuesta, columnas, filas);

	system ("pause>null");
	return 0;
}

\end{lstlisting}